\documentclass[xetex,serif]{beamer}
\usepackage{bibentry}
\usetheme{sanour}
\usepackage{xltxtra}

\XeTeXlinebreaklocale "th"

\defaultfontfeatures{Scale=1.4}



% Please start editing from this line 
\title{การเสนอผลงานด้วย Beamer ธีม ``เสนอ"}
\author{นายชอบ สไลด์สวย}
\institute[XUT]{มหาวิทยาลัยเทคโนโลยีซักแห่งหนึ่ง (XUT) }

\begin{document}
\begin{frame}
    \maketitle
\end{frame}



\begin{frame}{นิยามของคำว่า ``สไลด์สวย''}
ถึงแม้ว่าคำว่า ``สวย'' นั้นขึ้นอยู่กับตัวบุคคล เราคิดว่าความเรียบง่ายคือส่วนประกอบที่สำคัญของ \textbf{สไลด์สวย}

\bigskip 

นอกจากนั้นฟอนต์ก็เป็นส่วนสำคัญ เราเลือกใช้
ฟอนต์ \textbf{สุขุมวิท}  จาก \mbox{\textbf{บ. คัดสรรดีมาก}}
\end{frame}

\begin{frame}{การเน้นข้อความด้วยกล่อง}
	\begin{block}{กล่องทั่วไป / block}
	
	ในนี้จะใส่อะไรก็ได้โตแล้ว
	\end{block}
	
		\begin{exampleblock}{กล่องตัวอย่าง / exampleblock }

ดูคนนั้นเป็นตัวอย่างซิ 	
	\end{exampleblock}
	
	\begin{alertblock}{กล่องแจ้งเตือน / alertbox}
แต่ห้ามใส่ขาสั้นน่ะ
	\end{alertblock}

	
\end{frame}

\begin{frame}{สมการ}
	การใช้ $\LaTeX$ ทำให้ สามารถพิมพ์สมการได้อย่างสะดวก
\par 
	
		\begin{block}{เมื่อแอปเปิ้ลตกใส่หัว}
			\begin{align*}
				F = ma			
			\end{align*}

		\end{block}
\end{frame}

\begin{frame}{การป้องกันการตัดคำ}

ใช้คำสั่ง \textbackslash mbox เพื่อป้องกันการตัดคำไทย

\end{frame}

\begin{frame}{การแสดงผลด้วยลิสต์}
\begin{itemize}
	\item อันนี้แบบ itemize
	\item ไม่มีเลขนำ \begin{itemize}
	\item ย่อไปอีก
	\end{itemize}
\end{itemize}
	
\begin{enumerate}
	\item อันนี้แบบ enumerate
	\item มีเลขด้วย
\end{enumerate}
\end{frame}


\begin{frame}{การจัดทำคอลัมน์}

\begin{columns}
\begin{column}{0.5\textwidth}
นั่งตากลม ชมดาว พราวระยิบ
กระซิบหวาน ขานรับ กับดารา
\bigskip
ฝังคำหวาน ผ่านสายลม พรมใบหญ้า
มีความรัก ยิ้มรื่น ชื่นฉ่ำใจ



\end{column}
\begin{column}{0.5\textwidth}
เสียงกระซิบ สายลม พรมใบหญ้า
ปรารถนา รักหญ้า กว่าใครใคร

\bigskip

ให้อิจฉา สายลม พรมพัดไหว
ฉันยังไร้ คู่ชม ตรมอุรา


\end{column}
\end{columns}

	
\end{frame}

\begin{frame}{ลองอ้างอิง}

อันนี้ภาษาไทย \cite{phongpaichit1996} แล้วก็อันนี้ภาษาอังกฤษ \cite{4600388}

	\begin{alertblock}{ข้อจำกัด}
		ชื่อผู้เขียนภาษาไทยยังไม่ลองรับ
	\end{alertblock}	
\end{frame}


\begin{frame}[allowframebreaks]
        \frametitle{References}
        \bibliographystyle{amsalpha}
        \bibliography{references.bib}
\end{frame}


\end{document}